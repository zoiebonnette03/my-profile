\documentclass[10pt, letterpaper]{article}

% Packages:
\usepackage[
    ignoreheadfoot, % set margins without considering header and footer
    top=2 cm, % seperation between body and page edge from the top
    bottom=2 cm, % seperation between body and page edge from the bottom
    left=2 cm, % seperation between body and page edge from the left
    right=2 cm, % seperation between body and page edge from the right
    footskip=1.0 cm, % seperation between body and footer
    % showframe % for debugging 
]{geometry} % for adjusting page geometry
\usepackage{titlesec} % for customizing section titles
\usepackage{tabularx} % for making tables with fixed width columns
\usepackage{array} % tabularx requires this
\usepackage[dvipsnames]{xcolor} % for coloring text
\definecolor{primaryColor}{RGB}{0, 79, 144} % define primary color
\usepackage{enumitem} % for customizing lists
\usepackage{fontawesome5} % for using icons
\usepackage{amsmath} % for math
\usepackage[
    pdftitle={John Doe's CV},
    pdfauthor={John Doe},
    pdfcreator={LaTeX with RenderCV},
    colorlinks=true,
    urlcolor=primaryColor
]{hyperref} % for links, metadata and bookmarks
\usepackage[pscoord]{eso-pic} % for floating text on the page
\usepackage{calc} % for calculating lengths
\usepackage{bookmark} % for bookmarks
\usepackage{lastpage} % for getting the total number of pages
\usepackage{changepage} % for one column entries (adjustwidth environment)
\usepackage{paracol} % for two and three column entries
\usepackage{ifthen} % for conditional statements
\usepackage{needspace} % for avoiding page brake right after the section title
\usepackage{iftex} % check if engine is pdflatex, xetex or luatex

% Ensure that generate pdf is machine readable/ATS parsable:
\ifPDFTeX
    \input{glyphtounicode}
    \pdfgentounicode=1
    % \usepackage[T1]{fontenc} % this breaks sb2nov
    \usepackage[utf8]{inputenc}
    \usepackage{lmodern}
\fi



% Some settings:
\AtBeginEnvironment{adjustwidth}{\partopsep0pt} % remove space before adjustwidth environment
\pagestyle{empty} % no header or footer
\setcounter{secnumdepth}{0} % no section numbering
\setlength{\parindent}{0pt} % no indentation
\setlength{\topskip}{0pt} % no top skip
\setlength{\columnsep}{0cm} % set column seperation
\makeatletter
\let\ps@customFooterStyle\ps@plain % Copy the plain style to customFooterStyle
\patchcmd{\ps@customFooterStyle}{\thepage}{
    \color{gray}\textit{\small John Doe - Page \thepage{} of \pageref*{LastPage}}
}{}{} % replace number by desired string
\makeatother
\pagestyle{customFooterStyle}

\titleformat{\section}{\needspace{4\baselineskip}\bfseries\large}{}{0pt}{}[\vspace{1pt}\titlerule]

\titlespacing{\section}{
    % left space:
    -1pt
}{
    % top space:
    0.3 cm
}{
    % bottom space:
    0.2 cm
} % section title spacing

\renewcommand\labelitemi{$\circ$} % custom bullet points
\newenvironment{highlights}{
    \begin{itemize}[
        topsep=0.10 cm,
        parsep=0.10 cm,
        partopsep=0pt,
        itemsep=0pt,
        leftmargin=0.4 cm + 10pt
    ]
}{
    \end{itemize}
} % new environment for highlights

\newenvironment{highlightsforbulletentries}{
    \begin{itemize}[
        topsep=0.10 cm,
        parsep=0.10 cm,
        partopsep=0pt,
        itemsep=0pt,
        leftmargin=10pt
    ]
}{
    \end{itemize}
} % new environment for highlights for bullet entries


\newenvironment{onecolentry}{
    \begin{adjustwidth}{
        0.2 cm + 0.00001 cm
    }{
        0.2 cm + 0.00001 cm
    }
}{
    \end{adjustwidth}
} % new environment for one column entries

\newenvironment{twocolentry}[2][]{
    \onecolentry
    \def\secondColumn{#2}
    \setcolumnwidth{\fill, 4.5 cm}
    \begin{paracol}{2}
}{
    \switchcolumn \raggedleft \secondColumn
    \end{paracol}
    \endonecolentry
} % new environment for two column entries

\newenvironment{header}{
    \setlength{\topsep}{0pt}\par\kern\topsep\centering\linespread{1.5}
}{
    \par\kern\topsep
} % new environment for the header

\newcommand{\placelastupdatedtext}{% \placetextbox{<horizontal pos>}{<vertical pos>}{<stuff>}
  \AddToShipoutPictureFG*{% Add <stuff> to current page foreground
    \put(
        \LenToUnit{\paperwidth-2 cm-0.2 cm+0.05cm},
        \LenToUnit{\paperheight-1.0 cm}
    ){\vtop{{\null}\makebox[0pt][c]{
        \small\color{gray}\textit{Last updated in September 2024}\hspace{\widthof{Last updated in September 2024}}
    }}}%
  }%
}%

% save the original href command in a new command:
\let\hrefWithoutArrow\href

% new command for external links:
\renewcommand{\href}[2]{\hrefWithoutArrow{#1}{\ifthenelse{\equal{#2}{}}{ }{#2 }\raisebox{.15ex}{\footnotesize \faExternalLink*}}}


\begin{document}
    \newcommand{\AND}{\unskip
        \cleaders\copy\ANDbox\hskip\wd\ANDbox
        \ignorespaces
    }
    \newsavebox\ANDbox
    \sbox\ANDbox{}

    \placelastupdatedtext
    \begin{header}
        \textbf{\fontsize{24 pt}{24 pt}\selectfont Zoie Bonnette}

        \vspace{0.3 cm}

        \normalsize
        \mbox{{\color{black}\footnotesize\faMapMarker*}\hspace*{0.13cm}Lubbock, TX}%
        \kern 0.25 cm%
        \AND%
        \kern 0.25 cm%
        \mbox{\hrefWithoutArrow{mailto:bonnettezoie@gmail.com}{\color{black}{\footnotesize\faEnvelope[regular]}\hspace*{0.13cm}bonnettezoie@gmail.com}}%
        \kern 0.25 cm%
        \AND%
        \kern 0.25 cm%
        \mbox{\hrefWithoutArrow{tel:+1-949-449-3192}{\color{black}{\footnotesize\faPhone*}\hspace*{0.13cm}(949) 449-3192}}%
        \kern 0.25 cm%
        \AND%
        \kern 0.25 cm%
        \mbox{\hrefWithoutArrow{https://zoiebonnette03.github.io/my-profile/}{\color{black}{\footnotesize\faLink}\hspace*{0.13cm}zoiebonnette03.github.io}}%
        \kern 0.25 cm%
        \AND%
        \kern 0.25 cm%
        \mbox{\hrefWithoutArrow{https://linkedin.com/in/zoie-bonnette}{\color{black}{\footnotesize\faLinkedinIn}\hspace*{0.13cm}zoie-bonnette}}%
        \kern 0.25 cm%
        \AND%
        \kern 0.25 cm%
        \mbox{\hrefWithoutArrow{https://github.com/zoiebonnette03}{\color{black}{\footnotesize\faGithub}\hspace*{0.13cm}zoiebonnette03}}%
    \end{header}

    \vspace{0.3 cm - 0.3 cm}


    \section{Summary}



        
        \begin{onecolentry}
           Bachelor of Science in Computer Science candidate, and experience developing front-end web components using Angular, collaborating with software teams, and testing API controllers in C\#. Passionate about using software development as a problem-solving tool to drive improvements within the public sector. 

        \end{onecolentry}

        \vspace{0.2 cm}


    \section{Education}

        
        \begin{twocolentry}{
            
            
        \textit{Sept 2021 – May 2025}}
            \textbf{Texas Tech University}

            \textit{BS in Computer Science, minor in Mathematics}
        \end{twocolentry}

        \vspace{0.10 cm}
        \begin{onecolentry}
            \begin{highlights}
                \item GPA: 3.9/4.0 (\href{https://example.com}{a link to somewhere})
                \item \textbf{Coursework:} Computer Architecture, Comparison of Learning Algorithms, Computational Theory
            \end{highlights}
        \end{onecolentry}



    
    \section{Experience}



        
        \begin{twocolentry}{
        \textit{Lubbock, TX}    
            
        \textit{September 2024 – Present}}
            \textbf{Software Engineer}
            
            \textit{Tyler Technologies}
        \end{twocolentry}

        \vspace{0.10 cm}
        \begin{onecolentry}
            \begin{highlights}
                \item Developed an API controller and created endpoints to handle data transactions between the client and server
                \item Designed  and implemented front-end components, integrating API endpoints, enabling data entry and updates to the SQL database 
                \item Enhanced data presentation and functionality by redesigning tables with AG Grid, improving sorting capabilities and user experience 
            \end{highlights}
        \end{onecolentry}


        \vspace{0.2 cm}

        \begin{twocolentry}{
        \textit{Lubbock, TX}    
            
        \textit{June 2024 – Aug 2024}}
            \textbf{Software Engineer Intern}
            
            \textit{Tyler Technologies}
        \end{twocolentry}

        \vspace{0.10 cm}
        \begin{onecolentry}
            \begin{highlights}
                \item Implemented a mobile-friendly “next” button with enable/ disable functionality for a web application using Angular and BroswerStack for cross-browser testing

                \item Developed and executed unit tests in C\# using Moq library, xUnit framework, and Visual Studios debugging tools to ensure code quality and functionality 
                \item Built a real-time “user is typing…” indicator with animated ellipses using event listeners in Angular 
                \item Participated in recurring Scrum meetings (stand-up, sprint planning, sprint reviews, retrospectives), gaining hands-on experience in collaborative software development practices and team workflow 

               
            \end{highlights}
        \end{onecolentry}


    
    \section{Projects}



        
        \begin{twocolentry}{
            
            
        \textit{\href{https://github.com/zoiebonnette03/Multithreading-Matrix-Processing}{github.com/Multithread}}}
            \textbf{Multithreading Matrix Processing}
        \end{twocolentry}

        \vspace{0.10 cm}
        \begin{onecolentry}
            \begin{highlights}
                \item Processed matrix data using Conway’s Game of Life rules on Texas Tech’s High Performance Computing Center, utilizing multi-threading for efficient computation for large-scale simulations 

                \item Tools Used: Python, TTU HPCC
            \end{highlights}
        \end{onecolentry}


        \vspace{0.2 cm}

        \begin{twocolentry}{
            
            
        \textit{\href{https://github.com/zoiebonnette03/Recursive-Descent-Parser}{github.com/Parser}}}
            \textbf{Recursive Descent Parser}
        \end{twocolentry}

        \vspace{0.10 cm}
        \begin{onecolentry}
            \begin{highlights}
                \item Developed a lexeme parser to analyze and validate input phrases or words based on provided rules using efficient parsing algorithms while ensuring no ambiguity 
                \item Tools Used: C, Makefile, VS
            \end{highlights}
        \end{onecolentry}


        \vspace{0.2 cm}

        \begin{twocolentry}{
            
            
         \textit{\href{https://github.com/zoiebonnette03/red-black-treehttps://github.com/zoiebonnette03/red-black-tree}{github.com/RBTree}}}
            \textbf{Red-Black Tree}
        \end{twocolentry}

        \vspace{0.10 cm}
        \begin{onecolentry}
            \begin{highlights}
                \item Developed a self-sorting red-black tree, integrating algorithms to balance the tree during insertions and deletions
\item Optimizes search time by inserting numbers into a self-sorting stack 
                \item Tools Used: C++
            \end{highlights}
        \end{onecolentry}



    
    \section{Technologies}



        
        \begin{onecolentry}
            \textbf{Languages:} C++, Python, Java, C\#, SQL, HTML, SCSS, TypeScript, JavaScript/ JSX 
        \end{onecolentry}

        \vspace{0.2 cm}

        \begin{onecolentry}
            \textbf{Technologies:} Git, Jira, Docker, VS/ VS Code, SQL SMS, AWS, Figma, Expo
        \end{onecolentry}


    

\end{document}